\documentclass[]{article}

\usepackage[utf8]{inputenc}
\usepackage{listings}
\setlength{\oddsidemargin}{-0.25 in}
\setlength{\evensidemargin}{-0.25 in}
\setlength{\topmargin}{-0.9 in}
\setlength{\textwidth}{7.0 in}
\setlength\parindent{24pt}
\setlength{\textheight}{9.0 in}
\setlength{\headsep}{0.75 in}
\usepackage{indentfirst}
\setlength{\parskip}{0.1 in}
\newcommand{\nl}{\\ \\}

\begin{document}
\title{Programming Assignment 2}
\author{Alexander Solis and Hana Kim}
\date{March 25, 2015}
\maketitle

\section{Analytical Analysis}
In this section, we explain how we determine the cross-over point between Strassen's algorithm and the traditional matrix multiplication method mathematically.  The cross-over point is the value of the dimension, $n_0$, at which Strassen's algorithm should stop running on the multiplication of two matrices of type $nxn$.  After $n_0$, the conventional multiplication should be utilized.  

In order to determine the cross-over point $n_0$, one needs to find the point at which running Strassen's algorithm on $n_0$, and subsequently the conventional method on the remaining two matrices is exactly equal to to running the traditional matrix method on the two matrices of size $n_0$.  In order to do this, one needs to: 

\begin{enumerate}
\item Determine the closed-form equation for the number of operations that the conventional method takes 
\item Determine the exact recurrence relation for the number of operations that Strassen's algorithm takes
\item Compare these two equations and solve for the appropriate n, such that the two are equal
\end{enumerate}

\subsection{The Conventional Matrix Method}
The conventional matrix method two matrices $A$ and $B$ and multiplies every row value of matrix A  by every column value of matrix B in order to derive their product, matrix C, which is of the same dimension as matrix A and B (assuming that the dimensions are equal and square).  In order to derive the cost for multiplying two matrices each of size $n $x$ n$, first note that there are $n^2$ values in each matrix.  Determine the cost of obtaining the value of a single value in the result matrix, $C[0][0]$.  In order to get this value, one utilizes the method learned in his linear algebra class, $C[1][1]$: $$C[1][1] = a[1][1]*b[1][1] + a[1][2]*b[2][1] + a[1][3]*b[3][1] + .... + a[1][n]*b[n][1]$$

As one can see from this equation, getting this value takes $n$ multiplication operations and $n-1$ additions, for a grand total of $n+(n-1) = 2n -1$ operations.   Because there are necessarily $n^2$ values in the result matrix, matrix C, one needs to perform these $n+(n-1)$ $n^2$ times, which means that the total operation cost of multiplying two matrices via the conventional method is $n^2(2n-1) = 2n^3 - n^2$ operations for the conventional matrix multiplication method.  

\subsection{Recurrence Equation for Strassen's Method}
Next, one needs to find a proper recurrence relation to describe the operation cost of Strassen's method. 
Let $m(n) =$ number of multiplications used to multiply two $nxn$ matrices using Strassen's method.  Strassen's method solves for seven intermediate values of $x$ after it divides a matrix into four parts.  These equations look like this: $$x_1 = (a_{11} +a_{22}) *(b_{11} + b_{22})$$ $$ x_2 = (a_{21} +a_{22}) * b_{11}$$ $$x_3 = a_{11}  * (b_{21} +b_{22}). $$ $$x_4 = a_{22} *(b_{21} - b_{11})$$ $$x_5 = (a_{11} +a_{12}) * b_{22}$$ $$x_6 = (a_{21} - a_{11}) * (b_{11} +b_{12}) $$ $$ x_7 = (a_{12} - a_{22}) * (b_{21} - b_{22})$$   

As one can see from these seven calculations, each call of Strassen's requires seven multiplication operations.  

Then, let $p(n) = $ the number of additions and subtractions required by Strassen's algorithm to multiply two matrices.  For each value in the product matrix, after the $x's$ described above are calculated, Strassen's solves for the four quadrants of the resultant matrix, $C$.  Two of these equations look like this: $$C_{21} = x_2 + x_4$$ $$C_{11} = x_1 + x_4 - x_5 + x_7$$.  $$C_{12} = x_3 + x_5$$ $$C_{22} = x_1 - x_2 + x_3 + x_6 $$

When the number of addition and subtraction operations in the above equations is counted (including the seven x equations), one finds that there are 18 such operations.  Because this operation has to be done for each value of the resultant matrix, they need to be done $n^2$ times.  Thus, there are $18n^2$ such operations per Strassen's call.  

In total, then, we can determine that the recurrence relation for Strassen's algorithm is $$T(n) = 7T(\lceil n/2 \rceil) + 18(\lceil n/2 \rceil )^2$$

Note that the ceiling values are placed around $n$ because one is not assuming that n is even or odd.  In the case that n is odd, then, one will want to pad that matrix with an extra dimension, as Strassen's does not work with odd-dimensioned matrices.  When we do our comparisons, we will need to analyze two distinct cases, one in which n is odd and the other when n is even. 

\subsection{Comparison}
Now, we want to find the value of $n_0$ at which, upon multiplying two $n$ x $n$ matrices initially using Strassen's algorithm, one stops running Strassen's algorithm and switches to the conventional method.  That is, $n_0$ should be the value of that dimension where running the conventional method on $n_0$ runs at the same run time as running Strassen's on $n_0$, then the conventional method on any subsequent matrix multiplication calculations required by the recursion.  

That is, we want to find $n_0$ such that: $$7((2\lceil n_0/2 \rceil)^3 -( \lceil n_0/2 \rceil)^2)) + 18(\lceil n_0/2 \rceil )^ 2 = 2n_0^3 - n_0^2  $$

We look at two cases, since we want to analyze it when $n$ is either even or odd:

\subsubsection{n is odd}
$$7(2(\frac{n+1}{2})^3 - (\frac{n+1}{2})^2) + 18(\frac{n+1}{2})^2 = 2n^3 -n^2


\end{document}